\documentclass[10pt]{article}
\usepackage{fullpage}
\usepackage[T1]{fontenc}

\begin{document}

\title{Embedded Development Tools}
\author{Toan Vuong and Sam Mansfield\\
        EECS149\\
        Lab 2}
\date{September 18th, 2013}
\maketitle

\section*{Introduction}
  This lab introduced us to the embedded development tools we can use to program the myRIO. We briefly went through the setup and deployment of a hello world app using the Microblaze processor in Xilinx SDK, using the myRIO processor using Eclipse, and using the myRIO processor using LabVIEW.
\section*{Analysis}
  \subsection*{3.1 Connect to and Configure myRIO}
    To setup and configure the myRIO was straightforward and had plenty of screenshots and documentation. This made the configuration very easy.
  \subsection*{3.2 Program MicroBlaze from Xilinx SDK}
    Using the Xilinx SDK was mostly straight forward. It was a little unclear that we had to unplug the usb cable from the myRIO and plug it into the JTAG. Besides that we didn't run into any troubles.
  \subsection*{3.3 Program the myRIO Processor from Eclipse}
    One issue we ran into while doing this lab is a typo with the myRIO\textunderscore PrintStatus(status) line, which should have been \textbf{M}yRIO\textunderscore PrintStatus(status). Additionally, we seemingly got an error that the symbol D0LED30 could not be found. However, upon looking at the console our program compiled correctly. We were also able to run the debugger without little trouble. Overall, it was rewarding to see our program access the LED.
  \subsection*{3.4 Program the myRIO Processor from LabVIEW}
    This lab was straight forward. We implemented everything from the lab by looking at the given diagrams. Seeing the data from the onboard accelerometer alter the LED states was cool. 
\section*{Conclusion}
    Overall, this lab succeeds in introducing us to the tools. A few minor issues as noted above, but nothing non-trivial. We enjoyed that we got to interact with tools to program our embedded device, though the labs themselves could use more depth. However, even with the knowledge we have now, we already can write programs that interact with the myRIO and MicroBlaze.
\end{document}
