\documentclass[10pt]{article}
\usepackage{fullpage}
\usepackage{listings}
\usepackage{color}

\definecolor{mygreen}{rgb}{0,0.6,0}
\definecolor{mygray}{rgb}{0.5,0.5,0.5}
\definecolor{mymauve}{rgb}{0.58,0,0.82}

\lstset{ %
  backgroundcolor=\color{white},   % choose the background color; you must add \usepackage{color} or \usepackage{xcolor}
  basicstyle=\footnotesize,        % the size of the fonts that are used for the code
  breakatwhitespace=false,         % sets if automatic breaks should only happen at whitespace
  breaklines=true,                 % sets automatic line breaking
  captionpos=b,                    % sets the caption-position to bottom
  commentstyle=\color{mygreen},    % comment style
  deletekeywords={...},            % if you want to delete keywords from the given language
  escapeinside={\%*}{*)},          % if you want to add LaTeX within your code
  extendedchars=true,              % lets you use non-ASCII characters; for 8-bits encodings only, does not work with UTF-8
  frame=single,                    % adds a frame around the code
  keepspaces=true,                 % keeps spaces in text, useful for keeping indentation of code (possibly needs columns=flexible)
  keywordstyle=\color{blue},       % keyword style
  language=Octave,                 % the language of the code
  morekeywords={*,...},            % if you want to add more keywords to the set
  numbers=left,                    % where to put the line-numbers; possible values are (none, left, right)
  numbersep=5pt,                   % how far the line-numbers are from the code
  numberstyle=\tiny\color{mygray}, % the style that is used for the line-numbers
  rulecolor=\color{black},         % if not set, the frame-color may be changed on line-breaks within not-black text (e.g. comments (green here))
  showspaces=false,                % show spaces everywhere adding particular underscores; it overrides 'showstringspaces'
  showstringspaces=false,          % underline spaces within strings only
  showtabs=false,                  % show tabs within strings adding particular underscores
  stepnumber=2,                    % the step between two line-numbers. If it's 1, each line will be numbered
  stringstyle=\color{mymauve},     % string literal style
  tabsize=2,                       % sets default tabsize to 2 spaces
  %title=\lstname                   % show the filename of files included with \lstinputlisting; also try caption instead of title
}

\begin{document}
  \title{Lab 4: Programming Embedded Systems}
  \author{Toan Vuong and Sam Mansfield\\
          EECS149}
  \date{September 24th, 2013}
  \maketitle

  \section*{Generating Tones in Microblaze}
    \begin{enumerate}
      \setcounter{enumi}{2}
      \item Configure a Periodic Timed Interrupt 
        \begin{enumerate}
          \item Provide the content of your main program loop and interrupt service routine.\\[1em]
        \end{enumerate}

      \item Starve the Processor
        \begin{enumerate}
          \item At what frequency does MicroBlaze exhibit erratic behaviour or become unresponsive? What behavior did you observe?\\[1em]
        \end{enumerate}

      \item Share your Feedback\\[1em]
    \end{enumerate}

  \section*{Program an ADC in MicroBlaze}
    \begin{enumerate}
      \setcounter{enumi}{2}
      \item Poll the ADC 
      \begin{enumerate}
        \item Provide the content of your main() program loop.\\[1em]
          \lstinputlisting[language=C, firstline=97, lastline=163]{../lab3_data/timedIO2_1.c}
            
        \item When configuring the ADC in the main() program loop, were there any steps that set the rate at which the ADC is polled, or does your code run as fast as possible?\\[1em]
          The way we configure the ADC in the main() loop, there is no way to set the polling. We are grabbing the data as soon as we can get it. We can control how often it gets printed, but as soon as we ask for the ADC data we wait until it is ready. It would be possible to setup some sort of timed for loop to determine how often we ask for data, but when the actual conversion is started we just wait until it is ready.
      \end{enumerate}

      \item Use Timed Interrupts to Poll the ADC
        \begin{enumerate}
          \item Provide the content of main() that configures the timer ISR, as well as the body of the ISR routine.\\[1em]
            \lstinputlisting[language=C, firstline=58, lastline=89]{../lab3_data/timedIO2_2.c}
            \lstinputlisting[language=C, firstline=101, lastline=140]{../lab3_data/timedIO2_2.c}
        \end{enumerate}

      \item Use ADC and Timed Interrupts to Read the ADC
        \begin{enumerate}
          \item Provide the content of your main() function needed to configure the timer and ADC ISRs, as well as the content of the timer and ADC ISRs.\\[1em]
            \lstinputlisting[language=C, firstline=60, lastline=151]{../lab3_data/timedIO2_3.c}
        \end{enumerate}

      \item Share your Feedback\\[1em]
        This lab was very helpful. The only thing that was confusing was that we were writing to pin 8, but measuring from pin 0. I think this has something to do with that the mask was wrong in the code give to us, but I would have preferred to fix the mask then to the hacky method of just measuring from pin 0.
    \end{enumerate}
 
  \section*{Conclusion}
    In this lab we learned how to use interrupts with the Microblaze processor. It was very helpful to go through all the steps of initialization, enabling, and acknowledgement. The only problems we had in this lab were correctly setting the bits, which is our own fault, but overall the concepts are very helpful and should help us implement interrupts in the future.
\end{document}
