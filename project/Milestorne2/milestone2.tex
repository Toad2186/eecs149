\documentclass[10pt]{article}

\begin{document}
  \title{2nd Milestone Update Report}
  \author{Sam Mansfield and Toan Vuong\\
          EECS149
          Mentor: Mark Oehlberg}
  \date{November 26th, 2013}
  
  \maketitle

  \section{Introduction}
    Our project is to create an 8 x 8 x 8 LED chandelier that is interactive. We completed a prototype of a strand. It is fully functional and more or less up to specification. Our next step is to start going into production mode and try to make as many strands as we can. Our main limitations are time. Initially we were considering making 64 individual strands, but now we think it is more realistic to create 16.
    
  \section{Hardware}
    \subsection{What we have completed}
      We completed our first strand prototype. This includes a strand template for the LED's, a somewhat complete method to heat shrink the strands, a PCB prototype, and a strain relief mechanism. This step was crucial for our final production run. Right now we are using two small 3.7V batteries to power our project. 95\% of our hardware is selected and at hand. W
    \subsection{What's next}
      To go into our next phase, we have to complete several tasks in terms of hardware. We are designing a PCB that we will print using the PCB mill in the Invention Lab. We have experience using Eagle and using the Invention Lab PCB Mill, so there won't be a learning curve. We are also experimenting with using "fuzzy velvet\footnote{http://www.robot-electronics.co.uk/htm/reducing_sidelobes_of_srf10.htm}" to reduce interference from neighboring strands. We are also looking into using one instead of two batteries and then using a step up voltage regulator to provide power to the 5V ultrasonic sensor.
  \section{Software}
    \subsection{What we have completed}
    
    \subsection{What is next}
\end{document}
